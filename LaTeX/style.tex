% Modelo LaTeX para o PSE NE 2025.
% Feito por Silas Henrique (silash35@gmail.com)
% Versão 1.0.0 (20/11/2025)

% --- Pacotes Diversos ---
\usepackage[utf8]{inputenc}
\usepackage[T1]{fontenc}
\usepackage[brazilian]{babel}
\usepackage{xkeyval}

% --- Pacotes gráficos e matemáticos ---
\usepackage{amsmath}       % Recursos avançados de matemática
\usepackage{esdiff}        % Facilita escrita de derivadas (\diff, \pdiff)
\usepackage[version=4]{mhchem} % Escrever fórmulas químicas e reações

% --- Bibliografia ---
\usepackage[style=authoryear, doi=false, url=false]{biblatex}
\usepackage{csquotes}      % Aspas inteligentes para citações

% --- Configurações ---
\usebackgroundtemplate{
  \includegraphics[page=2,width=\paperwidth,height=\paperheight]{background.pdf}
} % Plano de fundo
\usepackage[scaled=0.92]{helvet} % Fonte Helvetica

% --- Traduções ---
\renewcommand{\sectionname}{Seção}

% --- Cores personalizadas ---
\definecolor{PSEblue}{HTML}{364971} % azul principal
\definecolor{PSEgray}{HTML}{757575} % cinza para rodapé e detalhes

% --- Configurações do Beamer ---
\usetheme{Copenhagen} % Tema geral
\setbeamertemplate{navigation symbols}{} % remove ícones de navegação
\setbeamertemplate{headline}{}           % esconde barra de topo
\setbeamertemplate{caption}[numbered]    % legendas numeradas

\usecolortheme[named=PSEblue]{structure}

\AtBeginSection[]{\frame{\sectionpage}}  % slide automático de seção

% Capa personalizada
\setbeamertemplate{title page}{
  \begin{center}
    \vspace{5mm}
    {\huge\textbf{\textcolor{PSEblue}{\inserttitle}}}

    \vspace{4mm}
    \insertauthor

    \vspace{2mm}
    \insertinstitute

    \vspace{4mm}
    \insertdate
  \end{center}
}

% Título dos slides
\setbeamercolor{frametitle}{fg=PSEblue}
\setbeamerfont{frametitle}{size=\LARGE, series=\bfseries}
\setbeamertemplate{frametitle}{
  \vspace{5mm}
  \hspace{-5mm}
  \insertframetitle
  \par
}

% Rodapé com numeração simples
\setbeamertemplate{footline}{%
  \begin{beamercolorbox}[wd=\paperwidth,ht=2.5ex,dp=1ex,rightskip=0.5cm]{footline}
    \hfill
    {\small \color{PSEgray}\insertframenumber{} / \inserttotalframenumber}
  \end{beamercolorbox}
  \vspace*{4mm}
}
