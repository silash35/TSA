% Modelo LaTeX para o PSE NE 2025.
% Desenvolvido por Silas Henrique (silash35@gmail.com),
% com contribuições de Daniel Diniz Santana (daniel.diniz@ufba.br).
% Versão 1.1.0 (21/11/2025)

% --- Outros Pacotes ---
\usepackage[T1]{fontenc}
\usepackage[utf8]{inputenc}
\usepackage[brazil]{babel}

\usepackage{xkeyval}

% --- Pacotes gráficos e matemáticos ---
\usepackage{amsmath}       % Recursos avançados de matemática
\usepackage{esdiff}        % Facilita escrita de derivadas (\diff, \pdiff)
\usepackage[version=4]{mhchem} % Escrever fórmulas químicas e reações
\usepackage{tikz} % Desenhos

% --- Bibliografia ----
\usepackage[alf]{abntex2cite}
% Correção: força tamanho normal nas citações
%\renewcommand{\cite}[1]{\normalsize\oldcite{#1}}
%\let\oldcite\cite

% --- Configurações ---
\usebackgroundtemplate{
  \includegraphics[page=2,width=\paperwidth,height=\paperheight]{background.pdf}
} % Plano de fundo
\usepackage[scaled=0.92]{helvet} % Fonte Helvetica

% --- Cores personalizadas ---
\definecolor{PSEblue}{HTML}{364971} % azul principal
\definecolor{PSEgray}{HTML}{757575} % cinza para rodapé e detalhes

% --- Configurações do Beamer ---
\usetheme{Copenhagen} % Tema geral
\setbeamertemplate{navigation symbols}{} % remove ícones de navegação
\setbeamertemplate{headline}{}           % esconde barra de topo
\setbeamertemplate{caption}[numbered]    % legendas numeradas

\usecolortheme[named=PSEblue]{structure}

\AtBeginSection[]{\frame{\sectionpage}}  % slide automático de seção

% Capa personalizada
\setbeamertemplate{title page}{
  \begin{center}
    \vspace{5mm}
    {\huge\textbf{\textcolor{PSEblue}{\inserttitle}}}

    \vspace{4mm}
    \insertauthor

    \vspace{2mm}
    \insertinstitute

    \vspace{4mm}
    \insertdate
  \end{center}
}

\setbeamertemplate{section page}{
  \begin{centering}
    \vfill
    \begin{beamercolorbox}[rounded=true,shadow=false,center,
        wd=\textwidth, % largura da caixa
        sep=10pt,      % espaçamento interno
        colsep*=1pt,   % espaçamento da borda
      ]{section title}
      \usebeamerfont{section title}\insertsection
    \end{beamercolorbox}
    \vfill
  \end{centering}
}

% Título dos slides
\setbeamercolor{frametitle}{fg=PSEblue}
\setbeamerfont{frametitle}{size=\LARGE, series=\bfseries}
\setbeamertemplate{frametitle}{
  \vspace{5mm}
  \hspace{-5mm}
  \insertframetitle
  \par
}

% Rodapé com numeração simples
\setbeamertemplate{footline}{%
  \begin{beamercolorbox}[wd=\paperwidth,ht=2.5ex,dp=1ex,rightskip=0.5cm]{footline}
    \hfill
    {\small \color{PSEgray}\insertframenumber{} / \inserttotalframenumber}
  \end{beamercolorbox}
  \vspace*{4mm}
}
