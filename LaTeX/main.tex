\documentclass[aspectratio=169]{beamer}
% Modelo LaTeX para o PSE NE 2025.
% Desenvolvido por Silas Henrique (silash35@gmail.com),
% com contribuições de Daniel Diniz Santana (daniel.diniz@ufba.br).
% Versão 1.1.0 (21/11/2025)

% --- Outros Pacotes ---
\usepackage[T1]{fontenc}
\usepackage[utf8]{inputenc}
\usepackage[brazil]{babel}

\usepackage{xkeyval}

% --- Pacotes gráficos e matemáticos ---
\usepackage{amsmath}       % Recursos avançados de matemática
\usepackage{esdiff}        % Facilita escrita de derivadas (\diff, \pdiff)
\usepackage[version=4]{mhchem} % Escrever fórmulas químicas e reações
\usepackage{tikz} % Desenhos

% --- Bibliografia ----
\usepackage[alf]{abntex2cite}
% Correção: força tamanho normal nas citações
%\renewcommand{\cite}[1]{\normalsize\oldcite{#1}}
%\let\oldcite\cite

% --- Configurações ---
\usebackgroundtemplate{
  \includegraphics[page=2,width=\paperwidth,height=\paperheight]{background.pdf}
} % Plano de fundo
\usepackage[scaled=0.92]{helvet} % Fonte Helvetica

% --- Cores personalizadas ---
\definecolor{PSEblue}{HTML}{364971} % azul principal
\definecolor{PSEgray}{HTML}{757575} % cinza para rodapé e detalhes

% --- Configurações do Beamer ---
\usetheme{Copenhagen} % Tema geral
\setbeamertemplate{navigation symbols}{} % remove ícones de navegação
\setbeamertemplate{headline}{}           % esconde barra de topo
\setbeamertemplate{caption}[numbered]    % legendas numeradas

\usecolortheme[named=PSEblue]{structure}

\AtBeginSection[]{\frame{\sectionpage}}  % slide automático de seção

% Capa personalizada
\setbeamertemplate{title page}{
  \begin{center}
    \vspace{5mm}
    {\huge\textbf{\textcolor{PSEblue}{\inserttitle}}}

    \vspace{4mm}
    \insertauthor

    \vspace{2mm}
    \insertinstitute

    \vspace{4mm}
    \insertdate
  \end{center}
}

\setbeamertemplate{section page}{
  \begin{centering}
    \vfill
    \begin{beamercolorbox}[rounded=true,shadow=false,center,
        wd=\textwidth, % largura da caixa
        sep=10pt,      % espaçamento interno
        colsep*=1pt,   % espaçamento da borda
      ]{section title}
      \usebeamerfont{section title}\insertsection
    \end{beamercolorbox}
    \vfill
  \end{centering}
}

% Título dos slides
\setbeamercolor{frametitle}{fg=PSEblue}
\setbeamerfont{frametitle}{size=\LARGE, series=\bfseries}
\setbeamertemplate{frametitle}{
  \vspace{5mm}
  \hspace{-5mm}
  \insertframetitle
  \par
}

% Rodapé com numeração simples
\setbeamertemplate{footline}{%
  \begin{beamercolorbox}[wd=\paperwidth,ht=2.5ex,dp=1ex,rightskip=0.5cm]{footline}
    \hfill
    {\small \color{PSEgray}\insertframenumber{} / \inserttotalframenumber}
  \end{beamercolorbox}
  \vspace*{4mm}
}


\addbibresource{bibliography.bib}
\graphicspath{{figures/}}

% Presentation settings
\title[Aplicação de Inteligência Artificial na Modelagem de Ciclos de PSA]{Aplicação de Inteligência Artificial na Modelagem de Ciclos de Adsorção por Modulação de Pressão}

\author[Silas Araújo]{
  Silas Araújo\inst{1},
  Ana Santos\inst{1},
  Diego Tachy\inst{1},
  Pablo Santos\inst{1},
  Paulo Dantas\inst{1},
  Arioston Morais\inst{2},
  Márcio A.F. Martins\inst{1}
}

\institute{
  \inst{1} Universidade Federal da Bahia (UFBA) \\
  \inst{2} Universidade Federal da Paraíba (UFPB)
}

\begin{document}

{
\setbeamertemplate{footline}{}
\usebackgroundtemplate{
  \includegraphics[page=1,width=\paperwidth,height=\paperheight]{background.pdf}
}
\maketitle
}

\section{Introdução}

\begin{frame}{Adsorção por Modulação de Pressão}
  \begin{columns}
    \begin{column}{0.5\textwidth}
      O processo de Adsorção por Modulação de Pressão (PSA) é uma técnica amplamente utilizada para separação e captura de gases, operando por meio de ciclos compostos por etapas sequenciais:
      \begin{itemize}
        \item Pressurização (\textit{Feed});
        \item Adsorção (\textit{Adsorption});
        \item Desorção (\textit{Desorption});
        \item Purga (\textit{Purge}).
      \end{itemize}


    \end{column}
    \begin{column}{0.5\textwidth}
      \begin{figure}
        \centering
        \includegraphics[width=1\textwidth]{psa-ciclo.png}
        \caption{Representação do processo PSA utilizado \parencite{rebello_2022}.}
      \end{figure}
    \end{column}
  \end{columns}
\end{frame}

\begin{frame}
  \begin{block}{Objetivo}
    \begin{itemize}
      \item Utilizar diferentes estratégias de modelagem orientada a dados para representar a dinâmica do processo de adsorção em sistemas PSA;
      \item comparar os modelos em termos de acurácia e velocidade de simulação.
    \end{itemize}
  \end{block}

  \begin{block}{Relevância}
    \begin{itemize}
      \item O PSA destaca-se por não requerer solventes químicos e por sua flexibilidade;
      \item A simulação fenomenológica do PSA é computacionalmente custosa;
      \item O desenvolvimento de modelos substitutos mais leves e rápidos aumenta a viabilidade da tecnologia.
    \end{itemize}
  \end{block}
\end{frame}

\section{Tratamento dos dados}

\begin{frame}{Dados do gPROMS}
  \begin{figure}
    \centering
    \includegraphics[width=0.9\textwidth]{../figures/KPIs-by-time.png}
    \caption{Dados de uma simulação de 110 ciclos no gPROMS. As linhas verticais representam o fim de cada ciclo.}
  \end{figure}
\end{frame}

\begin{frame}{Tratamento dos dados}
  \begin{figure}
    \centering
    \includegraphics[width=1\textwidth]{../figures/KPIs-by-cycle.png}
    \caption{Entradas e saídas do PSA, agora por ciclo.}
  \end{figure}
\end{frame}

\section{Modelo Linear (ARX)}

\begin{frame}
  \begin{equation}
    % Definição das variáveis
    y(k) =
    \begin{bmatrix}
      \text{purity}_{\ce{H2}}(k)    \\
      \ce{H2}/\ce{CO}(k)            \\
      \text{purity}_{\ce{CO2}}(k)   \\
      \text{recovery}_{\ce{CO2}}(k) \\
      \text{productivity}(k)
    \end{bmatrix},
    \qquad
    u(k) =
    \begin{bmatrix}
      t_{\text{feed}}(k)  \\
      t_{\text{rinse}}(k) \\
      t_{\text{blow}}(k)  \\
      t_{\text{purge}}(k)
    \end{bmatrix}
  \end{equation}

  \begin{align}
    \varphi(k) & =
    \begin{bmatrix}
      y(k-1) \\
      u(k-1) \\
      u(k)
    \end{bmatrix}
    \\[1em]
    y(k)       & = \varphi(k)^\top \theta
  \end{align}
\end{frame}

\begin{frame}{Simulação do Modelo ARX}
  \begin{figure}
    \centering
    \includegraphics[width=1\textwidth]{../figures/comparision-ARX.png}
    \caption{Comparação entre os dados do gPROMS e as saídas do modelo ARX.}
  \end{figure}
\end{frame}

\input{content/4.FNN.tex}
\input{content/5.RNN.tex}
\section{Considerações finais}

\begin{frame}{Coisas para fazer até o PSE NE}
  \begin{itemize}
    \item Colocar slides explicando cada modelo utilizado.
    \item Realizar testes de velocidade entre os modelos.
    \item Gerar mais dados para melhorar o treinamento dos modelos.
    \item Testar outros tipos de modelos baseados em inteligência artificial (como Echo State Networks).
  \end{itemize}
\end{frame}


{
  \logo{
    \raisebox{-1cm}{
      \includegraphics[height=1.8cm, keepaspectratio]{prh.png}%
    }
    \hspace{0.1cm}
    \raisebox{-1cm}{
      \includegraphics[height=1.8cm, keepaspectratio]{ufba.png}%
    }
    \hspace{2.0cm}
  }
  \begin{frame}{Agradecimentos}
    \centering \large Os autores gostariam de agradecer a ANP no âmbito do PRH 41/UFBA, pelo suporte financeiro e apoio ao desenvolvimento deste trabalho.
  \end{frame}
}

\begin{frame}[allowframebreaks]{Bibliografia}
  \printbibliography
\end{frame}

\end{document}