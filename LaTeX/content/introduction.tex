\section{Introdução}

\begin{frame}{Mas, o que são PINNs?}
  \begin{columns}
    \begin{column}{0.5\textwidth}
      % Introduzidas por \citeauthor{raissi_2017_I} em \citedate{raissi_2017_I}, as PINNs são uma forma de integrar princípios físicos ao treinamento de redes neurais. Elas têm o potencial de:
      \begin{itemize}
        \item Acelerar simulações em relação aos métodos numéricos tradicionais;
        \item melhorar a qualidade das previsões em comparação com redes neurais convencionais.
      \end{itemize}
    \end{column}
    \begin{column}{0.5\textwidth}
      \begin{figure}
        \centering
        % \resizebox{\textwidth}{!}{\input{../common/figures/neural-network.tex}}
        \caption{Representação de uma Rede Neural Artificial (ANN).}
      \end{figure}
    \end{column}
  \end{columns}
\end{frame}

\begin{frame}
  \begin{block}{Objetivo}
    \begin{itemize}
      \item Utilizar diferentes estratégias de modelagem orientada a dados para representar a dinâmica do processo de adsorção em sistemas PSA; comparar os modelos em termos de acurácia e velocidade de simulação.
    \end{itemize}
  \end{block}

  \begin{block}{Relevância}
    \begin{itemize}
      \item O PSA destaca-se por não requerer solventes químicos e por sua flexibilidade;
      \item A simulação fenomenológica do PSA é computacionalmente custosa;
      \item O desenvolvimento de modelos substitutos mais leves e rápidos aumenta a viabilidade da tecnologia.
    \end{itemize}
  \end{block}
\end{frame}
