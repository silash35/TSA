\section{Introdução}

\begin{frame}{Adsorção por Modulação de Pressão}
  \begin{columns}
    \begin{column}{0.5\textwidth}
      O processo de Adsorção por Modulação de Pressão (PSA) é uma técnica amplamente utilizada para separação e captura de gases, operando por meio de ciclos compostos por etapas sequenciais:
      \begin{itemize}
        \item Pressurização (\textit{Feed});
        \item Adsorção (\textit{Adsorption});
        \item Desorção (\textit{Desorption});
        \item Purga (\textit{Purge}).
      \end{itemize}


    \end{column}
    \begin{column}{0.5\textwidth}
      \begin{figure}
        \centering
        \includegraphics[width=1\textwidth]{psa-ciclo.png}
        \caption{Representação do processo PSA utilizado \parencite{rebello_2022}.}
      \end{figure}
    \end{column}
  \end{columns}
\end{frame}

\begin{frame}
  \begin{block}{Objetivo}
    \begin{itemize}
      \item Utilizar diferentes estratégias de modelagem orientada a dados para representar a dinâmica do processo de adsorção em sistemas PSA;
      \item comparar os modelos em termos de acurácia e velocidade de simulação.
    \end{itemize}
  \end{block}

  \begin{block}{Relevância}
    \begin{itemize}
      \item O PSA destaca-se por não requerer solventes químicos e por sua flexibilidade;
      \item A simulação fenomenológica do PSA é computacionalmente custosa;
      \item O desenvolvimento de modelos substitutos mais leves e rápidos aumenta a viabilidade da tecnologia.
    \end{itemize}
  \end{block}
\end{frame}
