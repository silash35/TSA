\section{Formulação do MPC}

\begin{frame}[allowframebreaks]
  \begin{equation}
    \begin{aligned}
      \min_{\Delta u_k} \quad &
      \sum_{i=1}^{N_p}
      \left( y(k+i) - y_{SP} - \nu \right)^{T}
      Q
      \left( y(k+i) - y_{SP} - \nu \right)
      \\[4pt]
                              & \quad +
      \sum_{i=1}^{N_c}
      \Delta u(k+i)^{T}\,
      R\,
      \Delta u(k+i)
      \\[6pt]
      \text{s.a.:} \quad
                              & y(k+i)
      = A\,y(k+i-1)
      + B
      \begin{bmatrix}
        u(k+i-1) \\
        u(k+i)
      \end{bmatrix},
      \qquad i = 1,\dots,N_p,
      \\[10pt]
                              &
      u(k+i)=u(k)+\displaystyle\sum_{j=1}^{i}\Delta u(k+j),\qquad i=1,\dots,N_c,
      \\[14pt]
                              &
      u_{\min} \le u(k+i) \le u_{\max},
      \qquad i = 1,\dots,N_c.
    \end{aligned}
  \end{equation}

  O termo integrador é atualizado por:
  \[
    \nu(k)=\nu(k-1)+S\,\bigl(y(k-1)-y_{SP}(k)\bigr),
  \]

  e o vetor de decisão é:
  \[
    \Delta u_k =
    \begin{bmatrix}
      \Delta u(k)^{T}   &
      \Delta u(k+1)^{T} &
      \dots             &
      \Delta u(k+N_c-1)^{T}
    \end{bmatrix}^{T}.
  \]

  $y(k)$ é o vetor de saídas;
  $u(k)$ o vetor de entradas;
  $A$ e $B$ são as matrizes identificadas do sistema;
  $N_p$ e $N_c$ são os horizontes de predição e de controle;
  $Q$ e $R$ são os pesos da saída e da variação de controle;
  $S$ representa os pesos do integrador;
  $u_{\min}$ e $u_{\max}$ são os limites das variáveis manipuladas.
\end{frame}
